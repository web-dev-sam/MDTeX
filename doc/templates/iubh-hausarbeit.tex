\documentclass[11pt,oneside,titlepage,listof=totoc,bibliography=totoc]{scrartcl}

\newcommand{\myAutor}{{{ AUTHOR }}}
\newcommand{\myTitel}{{{ TITLE }}}
\newcommand{\myBetreuer}{{{ TUTOR }}}
\newcommand{\myLehrveranstaltung}{{{ COURSE }}}
\newcommand{\myMatrikelNr}{{{ MATRIKEL }}}
\newcommand{\myAbgabeDatum}{\today}
\newcommand{\myHochschulName}{IU Internationale Hochschule}
\newcommand{\myHochschulStandort}{Bad Honnef}
\newcommand{\myStudiengang}{{{ STUDY }}}
\newcommand{\myThesisArt}{{{ TYPE }}}
\newcommand{\myAkademischerGrad}{{{ DEGREE }}}

\usepackage[utf8]{luainputenc}
\usepackage[ngerman]{babel}
\usepackage[babel,german=quotes]{csquotes}
\usepackage{fancyhdr}
\usepackage{fancybox}
\usepackage[a4paper, left=2cm, right=2cm, top=2cm, bottom=2cm]{geometry}
\usepackage{setspace}
\usepackage{fontspec}
\usepackage{graphicx}
\usepackage{colortbl}
\usepackage[capposition=top]{floatrow}
\usepackage{array}
\usepackage{float}
\usepackage{footnote}
\usepackage[singlelinecheck=false, labelfont=bf, font=bf]{caption}
\usepackage{caption}
\usepackage{enumitem}
\usepackage{amssymb}
\usepackage{mathptmx}
\usepackage{courier}
\usepackage{amsmath}
\usepackage[table]{xcolor}
\usepackage{marvosym}
\usepackage{ragged2e}
\usepackage[hang,multiple]{footmisc}
\usepackage[all]{nowidow}
\usepackage{epstopdf}
\usepackage{nicefrac}
\usepackage{multirow}
\usepackage{rotating}
\usepackage{mdwlist}
\usepackage{tabularx}
\usepackage{listings}
\usepackage{url}
\usepackage{hyphsubst}
\usepackage{acronym}
\usepackage[hyperfootnotes=false]{hyperref}

\setlength{\parindent}{0pt}
\setlength{\parskip}{0.8em plus 0.5em minus 0.3em}
\setlength{\marginparwidth}{2cm}
\setlength{\footnotemargin}{1em}
\renewcommand{\baselinestretch}{1.5}
\setstretch{1.5}
\setmainfont{Arial}
\lstdefinelanguage{JavaScript}{
	keywords={break, super, case, extends, switch, catch, finally, for, const, function, try, continue, if, typeof, debugger, var, default, in, void, delete, instanceof, while, do, new, with, else, return, yield, enum, let, await},
	keywordstyle=\color{blue}\bfseries,
	ndkeywords={class, export, boolean, throw, implements, import, this, interface, package, private, protected, public, static},
	ndkeywordstyle=\color{darkgray}\bfseries,
	identifierstyle=\color{black},
	sensitive=false,
	comment=[l]{//},
	morecomment=[s]{/*}{*/},
	commentstyle=\color{purple}\ttfamily,
	stringstyle=\color{red}\ttfamily,
	morestring=[b]',
	morestring=[b]"
}
\lstset{
	%language=JavaScript,
	numbers=left,
	numberstyle=\tiny,
	numbersep=5pt,
	breaklines=true,
	showstringspaces=false,
	frame=l ,
	xleftmargin=5pt,
	xrightmargin=5pt,
	basicstyle=\ttfamily\scriptsize,
	stepnumber=1,
	keywordstyle=\color{blue},
  	commentstyle=\color{dkgreen},
  	stringstyle=\color{mauve}
}
\urlstyle{same}
\HyphSubstIfExists{ngerman-x-latest}{%
\HyphSubstLet{ngerman}{ngerman-x-latest}}{}
\sloppy
\pagestyle{fancy}
\fancyhf{}
\fancyfoot[C]{\thepage}
\setlength{\footskip}{1cm}
\renewcommand{\headrulewidth}{0pt}
\hypersetup{colorlinks=true, breaklinks=true, linkcolor=darkblack, citecolor=darkblack, menucolor=darkblack, urlcolor=darkblack, linktoc=all, bookmarksnumbered=false, pdfpagemode=UseOutlines, pdftoolbar=true}
\urlstyle{same}

\hypersetup{
    pdfinfo={
        Title={asdasd},
        Subject={asd},
        Author={asd},
        Build=1.1
    }
}

\begin{document}

\pagenumbering{Roman}
\newcolumntype{C}{>{\centering\arraybackslash}X}
\thispagestyle{empty}

\renewcommand{\symheadingname}{Symbolverzeichnis}
\newcommand{\abbreHeadingName}{Abkürzungsverzeichnis}
\newcommand{\headingNameInternetSources}{Internetquellen}
\newcommand{\AppendixName}{Anhang}

% TITLEPAGE
\begin{titlepage}
	\newgeometry{left=2cm, right=2cm, top=2cm, bottom=2cm}
	\begin{center}
    \includegraphics[width=2.3cm]{doc/assets/IULogo.png} \\
    \vspace{.5cm}
		\begin{Large}\textbf{\myHochschulName}\end{Large}\\
    \vspace{.5cm}
		\vspace{2cm}
    \begin{Large}\textbf{\myThesisArt}\end{Large}\\
    \vspace{.5cm}
    im Studiengang \myStudiengang
		\vspace{1.7cm}

		im Rahmen der Lehrveranstaltung\\
    \vspace{0.5cm}
		\begin{Large}{\myLehrveranstaltung}\end{Large}\\
		\vspace{1.8cm}
		über das Thema\\
    \vspace{0.5cm}
		\large{\textbf{\myTitel}}\\
		\vspace{2cm}
    von\\
    \vspace{0.5cm}
    \begin{Large}{\myAutor}\end{Large}\\
	\end{center}
	\normalsize
	\vfill
    \begin{tabular}{ l l }
        Tutor: & \myBetreuer\\
        Matrikelnummer: & \myMatrikelNr\\
        Abgabedatum: & \myAbgabeDatum
    \\
    \end{tabular}
\end{titlepage}
% TITLEPAGE

\addtocontents{toc}{\protect\enlargethispage{-20mm}}% Die Zeile sorgt dafür, dass das Inhaltsverzeichnisseite auf die zweite Seite gestreckt wird und somit schick aussieht. Das sollte eigentlich automatisch funktionieren. Wer rausfindet wie, kann das gern ändern.
\setcounter{tocdepth}{4}
\tableofcontents
\thispagestyle{empty}
\newpage
\clearpage

\setcounter{page}{0}
\thispagestyle{empty}
\listoffigures
\newpage
\clearpage

\setcounter{page}{0}

\section*{Abkürzungsverzeichnis}
\begin{acronym}[WYSIWYG]\itemsep0pt %der Parameter in Klammern sollte die längste Abkürzung sein. Damit wird der Abstand zwischen Abkürzung und Übersetzung festgelegt
{{ ACRONYMS }}
\end{acronym}
\newpage

\pagenumbering{arabic}


{{ DOCUMENT }}


\newpage
\addcontentsline{toc}{section}{Literaturverzeichnis}
\section*{Literaturverzeichnis}

\section*{Anhang}
\newpage

%\addcontentsline{toc}{section}{Anhang}
%\section*{Anhang}
\newpage

\end{document}
