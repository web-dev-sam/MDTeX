\documentclass[11pt,oneside,titlepage,listof=totoc,bibliography=totoc]{scrartcl}

\newcommand{\myAutor}{{{ AUTHOR }}}
\newcommand{\myTitel}{{{ TITLE }}}
\newcommand{\myBetreuer}{{{ TUTOR }}}
\newcommand{\myLehrveranstaltung}{{{ COURSE }}}
\newcommand{\myMatrikelNr}{{{ MATRIKEL }}}
\newcommand{\myAbgabeDatum}{\today}
\newcommand{\myHochschulName}{IU Internationale Hochschule}
\newcommand{\myHochschulStandort}{Bad Honnef}
\newcommand{\myStudiengang}{{{ STUDY }}}
\newcommand{\myThesisArt}{{{ TYPE }}}
\newcommand{\myAkademischerGrad}{{{ DEGREE }}}

\usepackage[ngerman]{babel}
\usepackage[babel,german=quotes]{csquotes}
\usepackage{fancyhdr}
\usepackage{fancybox}
\usepackage[a4paper, left=2cm, right=2cm, top=2cm, bottom=2cm]{geometry}
\usepackage{setspace}
\usepackage{fontspec}
\usepackage{graphicx}
\usepackage{array}
\usepackage{float}
\usepackage{footnote}
\usepackage[singlelinecheck=false, labelfont=bf, font=bf]{caption}
\usepackage{caption}
\usepackage{enumitem}
\usepackage{amssymb}
\usepackage{mathptmx}
\usepackage{amsmath}
\usepackage{xcolor}
\usepackage{marvosym}
\usepackage{ragged2e}
\usepackage[hang,multiple]{footmisc}
\usepackage[all]{nowidow}
\usepackage{nicefrac}
\usepackage{multirow}
\usepackage{mdwlist}
\usepackage{tabularx}
\usepackage{listings}
\usepackage{url}
\usepackage{hyphsubst}
\usepackage{acronym}
\usepackage[normalem]{ulem}
\usepackage[hyperfootnotes=true]{hyperref}

\definecolor{darkblack}{rgb}{0,0,0}
\definecolor{hellgrau}{rgb}{0.9,0.9,0.95}

% Font/Text Style
\setlength{\parindent}{0pt}
\setlength{\parskip}{0.8em plus 0.5em minus 0.3em}
\setlength{\marginparwidth}{2cm}
\setlength{\footnotemargin}{1em}
\renewcommand{\baselinestretch}{1.5}
\setstretch{1.5}
\setmainfont{Arial}
\setmonofont{Noto Sans Mono}[Scale=0.85]
\setkomafont{disposition}{\normalfont\bfseries}
\urlstyle{same}
\sloppy

% Hyphenation Rules for German
\HyphSubstIfExists{ngerman-x-latest}{%
\HyphSubstLet{ngerman}{ngerman-x-latest}}{}

% Code Style
\newcommand{\code}[1]{\ttfamily\colorbox{hellgrau}{#1}\normalfont}
\definecolor{codegreen}{rgb}{0,0.6,0}
\definecolor{codegray}{rgb}{0.5,0.5,0.5}
\definecolor{codepurple}{rgb}{0.58,0,0.82}
\definecolor{codebg}{rgb}{0.95,0.95,0.98}
\lstdefinestyle{codestyle}{
    backgroundcolor=\color{codebg},   
    commentstyle=\color{codegreen},
    keywordstyle=\color{magenta},
    numberstyle=\tiny\color{codegray},
    stringstyle=\color{codepurple},
    basicstyle=\ttfamily\scriptsize,
    breakatwhitespace=false,         
    breaklines=true,                 
    captionpos=b,                    
    keepspaces=true,                 
    numbers=left,  
	numberstyle=\tiny,                  
    numbersep=10pt,  
	xleftmargin=5pt,
	xrightmargin=5pt,
	framexleftmargin=5pt,
	framexrightmargin=5pt,
    showspaces=false,                
    showstringspaces=false,
    showtabs=false,                  
    tabsize=2
}
\lstset{style=codestyle}

% Header and Footer Style
\pagestyle{fancy}
\fancyhf{}
\fancyfoot[C]{\thepage}
\renewcommand{\headrulewidth}{0pt}
\setlength{\footskip}{1cm}

% Section without preceeding number
\newcommand*{\nsection}[1]{
    \addcontentsline{toc}{section}{\protect\nonumberline #1}
	\section*{#1}
}

% Hyper Setup
\hypersetup{colorlinks=true, breaklinks=true, linkcolor=darkblack, citecolor=darkblack, menucolor=darkblack, urlcolor=darkblack, linktoc=all, bookmarksnumbered=false, pdfpagemode=UseOutlines, pdftoolbar=true}
\hypersetup{
    pdfinfo={
        Title={\myTitel},
        Subject={\myTitel},
        Author={\myAutor},
        Build=1.1
    }
}

\begin{document}

\pagenumbering{Roman}
\newcolumntype{C}{>{\centering\arraybackslash}X}
\thispagestyle{empty}

% TITLEPAGE
\begin{titlepage}
	\newgeometry{left=2cm, right=2cm, top=2cm, bottom=2cm}
	\begin{center}
    \includegraphics[width=2.3cm]{doc/assets/IULogo.png} \\
    \vspace{.5cm}
		\begin{Large}\textbf{\myHochschulName}\end{Large}\\
    \vspace{.5cm}
		\vspace{2cm}
    \begin{Large}\textbf{\myThesisArt}\end{Large}\\
    \vspace{.5cm}
    im Studiengang \myStudiengang
		\vspace{1.7cm}

		im Rahmen der Lehrveranstaltung\\
    \vspace{0.5cm}
		\begin{Large}{\myLehrveranstaltung}\end{Large}\\
		\vspace{1.8cm}
		über das Thema\\
    \vspace{0.5cm}
		\large{\textbf{\myTitel}}\\
		\vspace{2cm}
    von\\
    \vspace{0.5cm}
    \begin{Large}{\myAutor}\end{Large}\\
	\end{center}
	\normalsize
	\vfill
    \begin{tabular}{ l l }
        Tutor: & \myBetreuer\\
        Matrikelnummer: & \myMatrikelNr\\
        Abgabedatum: & \myAbgabeDatum
    \\
    \end{tabular}
\end{titlepage}
% TITLEPAGE

\pagenumbering{Roman}

\addtocontents{toc}{\protect\enlargethispage{-20mm}}% Die Zeile sorgt dafür, dass das Inhaltsverzeichnisseite auf die zweite Seite gestreckt wird und somit schick aussieht. Das sollte eigentlich automatisch funktionieren. Wer rausfindet wie, kann das gern ändern.
\setcounter{tocdepth}{4}
\tableofcontents
\newpage

\setcounter{page}{1}
\listoffigures
\newpage

\nsection{Abkürzungsverzeichnis}
\begin{acronym}[WYSIWYG]\itemsep0pt %der Parameter in Klammern sollte die längste Abkürzung sein. Damit wird der Abstand zwischen Abkürzung und Übersetzung festgelegt
{{ ACRONYMS }}
\end{acronym}
\newpage

\pagenumbering{arabic}
\setcounter{page}{1}


{{ DOCUMENT }}

\newpage
\pagenumbering{Roman}
\setcounter{page}{3}

\nsection{Literaturverzeichnis}
literatur...


\newpage
\nsection{Anhang}
anhang...


\end{document}
